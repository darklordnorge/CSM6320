\section{Uncertainty}\index{Uncertainty}
A purely logical approach may not work very well for statements which include multiple uncertain factors(e.g. Road accident on your way to the airport? Road work? Flooding? Nazi Zombies?\\
A purely logical approach would state: "$A_8$ will get me there on time."\\
Considering uncertain factors this would change to:"If there's no accident on the bridge, and it does not rain, and my tires remain intact etc, THEN $A_8$ will get me there on time"\\

Reasons for that:\\
\begin{itemize}
\item Failure to enumerate exceptions, qualifications etc.
\item No complete theory for the domain
\item Lack of relevant facts, initial conditions and so on
\end{itemize}

\subsection{Probability for handling uncertainty}
Probabilistic provides a way of summarizing the uncertainty.\\
An example for that would be: conditional probabilities can be used to represent:\\
\begin{itemize}
\item Given the available evidence, $A_8$ will get me there on time with probability 0.8
\item Given the available evidence, $A_10$ will get me there on time with probability 0.9
\item Given the available evidence, $A_12$ will get me there on time with probability 0.99
\end{itemize}

\begin{table}[h]
\centering
\begin{tabular}{l|l|l}
\toprule
\textbf{Condition} & \textbf{Result} & \textbf{Probability}\\
\midrule
Given the available evidence & $A_8$ & 0.9 \\
Given the available evidence & $A_10$ & 0.99\\
Given the available evidence & $A_12$ & 0.999\\
\bottomrule
\end{tabular}
\caption{Table representation of the different plans}
\end{table}

Probability theory is a main tool for dealing with degrees of belief.\\

\subsubsection{Making decisions}
Suppose given the following statements:\\

\begin{table}[h]
\centering
\begin{tabular}{l l}
P($A_8$  gets me there on time | Condition = & 0.9\\
P($A_{10}$ gets me there on time | Condition = & 0.99\\
P($A_{12}$ gets me there on time | Condition = & 0.999\\
P($A_{24}$ gets me there on time | Condition = & 0.9999\\
\end{tabular}
\end{table}

Given that, what actions to choose?\\
\begin{itemize}
\item Depends on preferences, e.g. the length of the wait at the airport
\end{itemize}

Utility theory is used to represent and infer preferences.\\
Decision theory = utility theory + probability theory\\

\section{Probability}
Subjective or Bayesian probability:\\

\begin{itemize}
\item Probabilities relate propositions to one's own state of knowledge, e.g. $P(A_8 | no reported accidents) = 0.9$
\item But might be learned from past experience from past experiences of similar situations
\item Probabilities of propositions change with new evidence: e.g. $P(A_8|no reported accidents, leave at 5 am) = 0.95$
\end{itemize}

Sample point:\\
A set $Omega $- the sample space: $\omega \in \Omega$ is a sample point or atomic event, e.g. 6 possible rolls of a dice.\\
A probability model/space is a sample space with assigning a probability for every $\omega \in \Omega$ 

\begin{itemize}
\item $P(1) = P(2) = P(3) = P(4) = P(5) = P(6) = 1/6$
\item Property: $0 \leq P(\omega) \leq 1 $ for every $\omega \in \Omega$, $\sum_{\omega \in \Omega} P(\omega) = 1$
\end{itemize}

\subsection{Event}
An event \textit{A} is any subset of $\Omega$.\\

\begin{equation}
P(A) = \sigma_{\omega \in A}P(\omega)
\end{equation}

\begin{itemize}
\item An event is 'dice roll is less than 4'
\item The probability of the event happening is $P(dice roll < 4)=P(1)+P(2)+P(3)=1/2$
\end{itemize}

In AI's language, the sets are decribed by propositions:
\begin{itemize}
\item $\phi$ is "dice roll is less than 4" 
\item $P(\phi) = \Sigma_{\omega \in \phi} P(\omega)$
\end{itemize}

\subsection{Random variables}
A random variable $X:\Omega \rightarrow R$ is a function from sample points to some range, e.g., the reals or Boolean.\\
Let $X(\omega)$ be a Boolean variable to represent whether a dicing result $\omega$ is odd, then:\\  

$X(1) = true$\\
$X(2) = False$\\

But usually written in short \textit{Odd} $P(Odd)$

\subsection{Probability Distribution}
For a random variable \textit{X} taking values from $x_1, \dots, x_k$ probability distribution $P(X = x_i)$ is the probability of \textit{X} taking the value of $x_i$:\\

\begin{table}[h]
\centering
\begin{tabular}{l l l}
$P(odd = true)$ & = & $P(1) + P(3) + P(5)$ \\
 & = & $1/6 + 1/6 + 1/6$ \\
 & = & $1/2$ \\
 $P(Odd = false)$ & = & $P(2) + P(4) + P(6)$ \\
 & = & $1/6 + 1/6 + 1/6$\\
 & = & $1/2$
\end{tabular}
\end{table}

\subsection{Prior probability}
Prior or unconditional probabilities refer to degrees of belief in propositions in the absence of any other information\\

\begin{itemize}
\item $P(Cavity = true) = 0.1$
\item $P(Weather = Sunny) = 0.72$
\end{itemize}

Probability distribution gives values for all possible probabilities:\\
$P(Weather) = <0.72, 0.1, 0.08, 0.1> $normalized, i.e. sums to 1.\\

\begin{table}[h]
\centering
\begin{tabular}{l l}
$P(Weather = sunny)$ & = 0.72\\
$P(Weather = rain)$ & = 0.1\\
$P(Weather = cloudy)$ & = 0.08\\
$P(Weather = snow)$ & = 0.1
\end{tabular}
\end{table}

\subsection{Joint Probability distribution}
Joint Probability distribution for a set of r.v.s gives the probability of every atomic event on those r.v.s(i.e. every sample point)\\

\begin{table}[h]
\centering
\begin{tabular}{|l|l|l|l|l|}
\toprule
 & \multicolumn{2}{ |c| }{Toothache} &\multicolumn{2}{ |c| }{$\neg$ Toothache}\\
 \hline
 & catch & $\neg$catch & catch & $\neq$ catch\\ 
\midrule
 cavity & 0.108 & 0.012 & 0.072 & 0.008\\
 \hline
 $\neg$cavity & 0.016 & 0.064 & 0.144 & 0.576 \\
 \bottomrule
\end{tabular}
\end{table}