\section{Case Based Reasoning}

\begin{itemize}
\item Case law: made up of, and from, hundred and thousands of precedent cases
\item Principle: cases with similar facts should be treated in similar ways 
\item In cases where the parties disagree on what the law is: looking to past precedential decisions of relevant courts
\item If a similar dispute has been resolved in the past: following the reasoning used in the prior decision
\item If the current dispute is fundamentally distinct from all previous cases: creating new 
\item CBR: analogous to human experts solving a problem through employing their relevant past experience
\begin{itemize}
\item exemplar-based reasoning
\item instance-based reasoning
\item memory-based reasoning
\item case-based reasoning
\item analogy-based reasoning
\end{itemize}
\item if the new problem has some novel aspects, aspects need to be created
\end{itemize}

Domains that CBR Works well:
\begin{itemize}
\item Broad but shallow domain
\begin{itemize}
\item bot a single tree, but a forest of small trees
\item a number of loosely connected problems that must be dealt with
\item need different kinds of expertise
\end{itemize}
\item Experience, rather than theory, is the primary source of knowledge
\begin{itemize}
\item Many past examples of problems that occur
\item rather than having a deep understanding of the domain
\end{itemize}
\item solutions are reusable
\begin{itemize}
\item old solution is useful for a new problem
\end{itemize}
\end{itemize}

\section{Case-Based Reasoning System and 4R Cycle}
\begin{itemize}
\item Input: new problem
\item Output: a solution to the new problem
\item case base
\begin{itemize}
\item store cases(experience)
\end{itemize}
\end{itemize}

\subsection{4R Cycle}
\begin{itemize}
\item Retrieve: relevant cases, match most similar cases, retrieve solutions from theses cases
\item Reuse: solutions in stored cases
\item Revise: the retrieved solution(s) to reflect differences between new case and retrieved case(s)
\item Retain: new cases into database
\end{itemize}

\begin{itemize}
\item Observations define a new problem
\item not all features may be known
\end{itemize}

\subsubsection{New Problem vs Old Case}
\begin{itemize}
\item Compare similarity of each feature
\item Similarity by weighted average
\end{itemize}

\section{Design Case-Based Reasoning System}
\subsection{Case Representation}
A case in diagnosis represents one diagnostic situation, include two parts:\\
\begin{enumerate}
\item Features
\begin{enumerate}
\item symptoms
\item failure
\item feature values
\item repair strategies
\end{enumerate}
\end{enumerate}

\subsubsection{Match Case}
\begin{itemize}
\item Compare features and their value between the stored case and new problem
\item Nearest-neighbour matching algorithm
\end{itemize}

\subsection{Retrieval: Ranking}
\begin{itemize}
\item Possibly more than one case is matched
\item Among matched cases, ranking may be used to choose a case to reuse
\item If a matched case cannot provide the solution to the problem,. lower rank cases may be taken as the candidate for the problem
\item Ranking value will depend on observation time and cost
\begin{itemize}
\item Higher the rank $\longrightarrow$ better solution(cheaper, faster, etc.)
\item Ranking Observation cost, observation time and case frequency need to be considered
\end{itemize}
\end{itemize}

\subsection{Reuse/Revive}
\begin{itemize}
Adapt/repair old solutions
\end{itemize}

\subsection{Retain: Store new cases and stop reasoning}
\begin{itemize}

\end{itemize}
